\documentclass[11pt,a4paper]{ivoa}
\input tthdefs

\usepackage{todonotes}
\usepackage{supertabular}

\lstset{flexiblecolumns=true,showstringspaces=False,
  language={}}

\newcommand{\parexample}[1]{\noindent\quad\texttt{#1}}

\title{Simple Line Access Protocol}

\ivoagroup{Data Access Layer}

\author{Moreau, N.}
\author{Salgado, J.}
\author{Osuna, P.}
\author{Demleitner, M.}
\author{Guainazzi, M.}
\author{Dubernet, M.-L.}
\author{Tody, D.}

\editor{Moreau, N.}

\previousversion[https://ivoa.net/documents/SLAP/20101209/]{REC-1.0}

\begin{document}
\begin{abstract}
The Simple Line Access Protocol (SLAP) is an IVOA
\textit{{Data Access protocol}} which defines a protocol for retrieving
spectral lines coming from various \textit{{Spectral Line Data
Collections}}  through a uniform interface within the VO framework.
These lines can be either observed or theoretical and will be typically
used to identify emission or absorption features in astronomical
spectra.

It makes  use of the Simple Spectral Line Data Model SSLDM
\citep{todo:ssldm} to characterize spectral lines through the use of
uTypes \citep{TODO: they're still unspecified}. Physical quantities of
units are described by using the standard VO Units syntax
\citep{2014ivoa.spec.0523D}.

SLAP services can be registered in an IVOA \textit{Registry of Resources}
where they have a unique \textit{ResourceIdentifier}
\citep{2016ivoa.spec.0523D}
They will be described using the Simple Line Access extension
\citep{2017ivoa.spec.0530P} of VOResource to specify their
compliance level with the SLAP standard and the accepted query
parameters.

The SLAP interface is meant to be reasonably simple to implement by
service providers. A basic query will be done in a wavelength range for
the different services, using the \{lines\} resource. The service
returns a list of spectral lines formatted as a VOTable.  An
implementation of the service may support additional search parameters
(some which may be custom to that particular service) to more finely
control the selection of spectral lines. Fully compliant services will
also provide the \{species\} resource, returning an exhaustive list of
all the species for which lines are available.

The specification also describes how the search on extra parameters has
to be done, making use of the support provided by the Simple
Spectral Line Data Model (SSLDM).
\end{abstract}

\section*{Acknowledgments}

The authors acknowledge the comments from the DAL WG members, from the
IVOA members in general and from the VAMDC Consortium members.

\section*{Conformance-related definitions}

% CHANGE FROM .DOC: this replaces the "Requirements for complicance"
% section that had some additional (but probably unused) language

The words ``MUST'', ``SHALL'', ``SHOULD'', ``MAY'', ``RECOMMENDED'', and
``OPTIONAL'' (in upper or lower case) used in this document are to be
interpreted as described in IETF standard RFC2119 \citep{std:RFC2119}.

The \emph{Virtual Observatory (VO)} is a
general term for a collection of federated resources that can be used
to conduct astronomical research, education, and outreach.
The \href{http://www.ivoa.net}{International
Virtual Observatory Alliance (IVOA)} is a global
collaboration of separately funded projects to develop standards and
infrastructure that enable VO applications.

\section{Introduction}

This Simple Line Access Protocol (SLAP henceforth) specification makes
use of the work done in the Simple Spectral Lines Data Model (SSLDM
henceforth) definition, as the source of the abstract representation of
a spectral line. All the information collected in this document will be
used to homogenize the access to the different existing spectral line
databases.

Web services in the Virtual Observatory infrastructure share a common
interface, described in the Data Access Layer Interface 1.1
\citep{2017ivoa.spec.0517D}, document describing behaviors that must be
implemented by all services concerning for example:

\begin{itemize}
\item the use of VOTable for encoding search results,
\item the mechanism for handling errors, and
\item the retrieval of service metadata.
\end{itemize}

Following this specifications, the SLAP interface is in many ways
similar to that of SSAP \citep{2012ivoa.spec.0210T} and SIAP (v2.0)
\citep{2015ivoa.spec.1223D}.  However, SIAP and SSAP protocols are
two-step processes. In the first step, the VO client application
requests metadata from the server. These metadata include links to
images or spectra. In the second step, the VO client application
requests the data from the server. In the particular case of Simple Line
Access Services, \ service will be, in essence, a one-step process,
i.e., only the first request for metadata is needed and from that point
of view it looks more like the Simple Cone Search protocol
\citep{2008ivoa.specQ0222P}.

Even when no link to astronomical products is expected because of the
nature of the service, the SLAP metadata could contain reference links
to html pages, spectra files, spectral line profiles, etc.

\subsection{Role within the VO Architecture}

\begin{figure}
\centering

\includegraphics[width=0.9\textwidth]{role_diagram.pdf}
\caption{Architecture diagram for this document}
\label{fig:archdiag}
\end{figure}

Fig.~\ref{fig:archdiag} shows the role this document plays within the
IVOA architecture \citep{note:VOARCH}.


\section{Spectral line service type}
\label{RefHeadingToc5716682}

It is assumed that compliant spectral line services fall into one of two categories.

\begin{enumerate}
\item Observational line databases. Lines observed and identified in
real spectra collected by different instrument/projects.
\item Theoretical line databases. Servers containing theoretical
spectral lines will be included in this group.
\end{enumerate}

In both cases, the line description and the identification might be
already present in a scientific publication, which should always be
provided when available to guarantee the provenance of data.

This document describes standard query parameters for SLAP services.
Some SLAP services might make use of extra parameters, not cited in this
document, to support additional filtering and selection.

However, the theoretical line database services could make use of extra
parameters not cited in this document to filter out lines not expected
to be identified in an observed spectrum or to score the output lines
due to the application of physical models.

Examples:

\begin{itemize}
\item For the Atomic spectral line database from CD-ROM 23 of R. L. Kurucz
\citep{kuruz}, the absorption oscillator strength, $\log(gf)$, can be selected

\item For the NIST Atomic Spectra Database Lines \citep{nist}
as it is extracted from the Saha-LTE model, ``\textit{The level
populations are calculated according to the Boltzmann distribution
within each ion and Saha distribution between the ion stages. Thus, to
calculate the spectrum from a single ion, e.g., C I, only
}\textbf{T}\textbf{\textsubscript{e}}\textit{ is required, while for the
spectrum from several ions of the same element (e.g., C I-V),
}\textbf{N}\textbf{\textsubscript{e}}\textit{ must be defined as
well.}''
\end{itemize}

At the same time, for observational spectral line databases, some
project specific search parameters may be used.

Example:

In ISO Astronomical Spectroscopy Database (IASD),
the observation number parameter can be used to select only the lines
observed during this ISO satellite observation.

Since it is awkward to compile all such extra parameters in a document
such as this one, a general mechanism is described. As will be explained
later, the discovery of these extra parameters by VO client applications
or by the registry relies on an implementation of a VOSI-capabilities
endpoint \citep{2017ivoa.spec.0524G}.

\section{Use cases}
\label{RefHeadingToc5716683}

The SLAP protocol aims at covering some classic use cases, some of which
will be described below.

\subsection{Finding available lines for a given species}
\label{RefHeadingToc5716684}

In the context of a numerical code, one might need to integrate
extensive data related to one or several given species.  In that case,
he/she would query one or several SLAP services supporting the
CHEMICAL\_ELEMENT parameter. If the \{species\} endpoint is available, a
first step could be to ask the service the complete list of species in
order to know if data are available or not.

\subsection[Line identification]{Line identification}
\label{RefHeadingToc5716685}

A user visualizing a spectrum might want to identify some absorption or
emission lines. As wavelength is a mandatory query parameter of the
\{lines\} endpoint for any valid SLAP service, the user could
potentially query any existing service, providing the wavelength range
inside which lines must be searched.

This feature can also be implemented into spectrum visualization
applications, providing a line identification function form a graphical
user interface.

\subsection{Discovering the content of a database}
\label{RefHeadingToc5716686}

In the context of a graphical user interface (be it a web interface or a
local standalone application) able to query SLAP services, an
application might need to provide the list of species available in a
service, in which the user could choose what he requires. To achieve
this, the application can send a query to the \{species\} endpoint of a
service and build a graphical list object from the content of the
returned VOTable.

By caching locally this list, one might also be able to implement an
auto-completion feature if the user wants to manually enter the name of
a species.

\clearpage\section{Query interface}
\label{RefHeadingToc5716687}

The SLAP resources are synchronous web service resources that conform to
the DALI-sync description \citep{2017ivoa.spec.0517D}. For a DALI-sync
resource, the parameters for a request may be submitted using an HTTP
GET (query string) or POST action.

\subsection{Resources}
\label{RefHeadingToc5716688}

Publishers are free to name (set paths) resources whatever they wish;
clients will find resource paths thanks to the VOSI-capabilities
resource.  The table below shows the list of available resources for a
service and their mandatory status.

\begin{inlinetable}
\begin{tabular}{p{4.806cm}p{4.894cm}p{4.951cm}}
\sptablerule
\textbf{Resource type}&
\textbf{Resource name}&
\textbf{Required}\\
\sptablerule
 DALI-sync &
 \{lines\} &
 yes\\
 DALI-sync &
 \{species\} &
 no\\
 DALI-examples &
 /examples &
 no\\
 VOSI-availability &
 service specific &
 yes\\
 VOSI-capabilities &
 /capabilities &
 yes\\
\sptablerule
\end{tabular}
\end{inlinetable}

In order to distinguish between resources when a service is declared in
a registry, \{lines\} and \{species\} will have distinct standardID as
specified in sect~\ref{toc368}.

\subsection{Input parameters}
\label{RefHeadingToc5716694}

As specified in the DALI recommendation, parameter names are not
case sensitive; a SLAP service must treat upper-, lower-, and mixed-case
parameter names as equal. Parameter values are case sensitive.

Some query parameters are multi-valued which means multiple occurrences
of the parameter=value pairs as specified in the DALI recommendation are
permitted. The constraints from multiple occurrences of a parameter are
combined with a logical OR operator. The constraints from different
parameters are combined with a logical AND operator.

Query parameters for numeric fields are either scalar, accepting a
single floating point or integer value, or a range of values. Such range
values are encoded using the VOTable array serialization (space
separated).

Numeric intervals are pairs of numeric values (integer and
floating-point). For floating point intervals, special values for
positive and negative infinity may be used to specify open-ended
intervals. Finite bounding values are included in the interval.

For example, the interval [300,600] is:

300 600

The open-ended interval [300,infinity) (all values greater than or equal
to 300) is:

300 +Inf

The open-ended interval (-infinity,600] (all values less than or equal
to 600) is:

-Inf 600

The open-ended interval (-infinity, infinity) (all values) is:

-Inf +Inf


\section{\{lines\} resource}
\label{RefHeadingToc5716699}

The purpose of this resource is to allow users/clients to search in a
wavelength range for spectral lines. This resource MUST be implemented
by a SLAP service. The most basic query parameters will be the minimum
and maximum value for the wavelength range. Additional parameters may be
used to refine the search or to model physical scenarios.

\subsection{Required parameters}
\label{RefHeadingToc5716700}

A service must support the input parameters described in this section.
That means that the service must accept them as valid ones without
raising an error, and the parameters must be properly used to constrain
the query.

\subsubsection[Wavelength]{Wavelength}
\label{RefHeadingToc5716701}

The service \textbf{MUST} support the \textbf{WAVELENGTH} parameter, to
specify the wavelength spectral range, to be specified in meters. This
wavelength range will be interpreted as the wavelength in the vacuum of
the transition originating in the line
({ucd={\textquotedbl}em.wl{\textquotedbl}};
utype=''\textbf{ssldm:Line.wavelength.value}'').

As the units in the spectral line database could be different than
meters, the service will need to translate from the selected units
(meters) to the internal ones. The selection of one type of units (in
this case the SI unit meters) will help to unify access to different
spectral line databases, even when in some cases, the unit selected
(meter) may not be the best one for the range on interest.

Example

To query for spectral lines in the wavelength range between 5.1 and 5.6
micrometers:

\parexample{WAVELENGTH=5.1E-6 5.6E-6}

\subsubsection{RESPONSEFORMAT}
\label{RefHeadingToc5716702}

The RESPONSEFORMAT parameter, defined in the DALI standard, is used so
the client can specify the format of the response. The possible values
in the context of a SLAP service are:

\begin{inlinetable}
\begin{tabular}{p{4.88cm}p{4.88cm}p{4.905cm}}
\sptablerule
\textbf{Table type}&
\textbf{Media type}&
\textbf{Short form}\\
VOTable&
application/x-votable+xml &
votable\\
VOTable &
text/xml &
votable\\
\sptablerule
\end{tabular}
\end{inlinetable}

If the parameter is not specified by the client in the query or uses the ``votable'' short form, the returned VOTABLE
use a base MIME-type of application/x-votable+xml (as specified in section 3.4.3 of the DALI specification).

Return a result with a MIME-type of text/xml:

\parexample{RESPONSEFORMAT=text/xml}

\subsubsection{MAXREC}
\label{RefHeadingToc5716703}

The MAXREC parameter is defined in DALI and allows the client to
limit the number or records in the response. A service implementation
may also impose default and maximum values for this limit. However the
limit is determined, if the output is truncated due to the limit, the
server must indicate this using an overflow (as described in section
4.4.1 of the DALI specification) indicator except in the special
case of MAXREC=0, where the service respond with metadata-only (normal
output document with no records).

Display only 100 lines:

\parexample{MAXREC=100}

\subsection{Non-compulsory parameters}
\label{RefHeadingToc5716704}

\subsubsection{Chemical element}
\label{RefHeadingToc5716705}

A service \textbf{MAY} have a search parameter called
\textbf{CHEMICAL\_ELEMENT}. This parameter would constrain the search to
the chemical element selected. A list of different chemical elements
could be queried by specifying this parameter multiple times.

(ucd=``phys.atmol.element''; utype=''\textbf{ssldm:Species.name''})

Atom can be specified exactly by symbol. Molecules can be specified by
conventional molecular name (CO2, CH4 {\dots}) which might not be
unique. The \{species\} endpoint is the recommended place to discover
the correct syntax of a given species in a particular service.

All lines related to the iron atom:

\parexample{CHEMICAL\_ELEMENT=Fe}

All lines related to the carbon dioxide molecule:

\parexample{CHEMICAL\_ELEMENT=CO2}

\subsubsection{Ion charge range}
\label{RefHeadingToc5716706}

A service \textbf{MAY} implement the parameter \textbf{ION\_CHARGE} to
specify the minimum and maximum charge of an ion.  It will look for the
ionized forms of the chemical element specified in CHEMICAL\_ELEMENT
parameter. If it is not defined, the restriction will be applied to all
the ions available. If several CHEMICAL\_ELEMENT values have been
provided, the ion charge values will be applied to each one of them.

The charge will be an integer value greater than 0 for ionized species
(positive charge), less than 0 for negative charge (excess electrons), 0
for neutral.

(ucd=''phys.atmol.ionization'';utype=''\textbf{ssldm:Species.ionCharge}'')

All lines related to Fe+ ion:

\parexample{CHEMICAL\_ELEMENT=Fe\&ION\_CHARGE=1}

\subsubsection{Energy level range}
\label{RefHeadingToc5716707}

\paragraph{Level energy}

A service \textbf{MAY} implement the parameter \textbf{LEVEL\_ENERGY} to
specify the minimum and maximum energy for any of the level of the
transition, to be expressed in Joules.

(ucd={\textquotedbl} phys.energy;phys.atmol.level''; utype=''\textbf{ssldm:Level.energy}'')

Energy of upper or lower level between 3.93E-18 and 3.94E-18 Joules:

\parexample{LEVEL\_ENERGY=3.93E-18 3.94E-18}

\paragraph{Lower level energy}

A service \textbf{MAY} implement the parameter
\textbf{LOWER\_LEVEL\_ENERGY} to specify the minimum and maximum energy
for the LOWER level of the transition, to be expressed in Joules.

(ucd={\textquotedbl} phys.energy;phys.atmol.level'';
utype=''\textbf{ssldm:Line.lowerLevel.energy.value''})

Energy of lower level between 3.93E-18 and 3.94E-18 Joules:

\parexample{LOWER\_LEVEL\_ENERGY=3.93E-18 3.94E-18}

\paragraph{Upper level energy}

A service \textbf{MAY }implement the parameter
\textbf{UPPER\_LEVEL\_ENERGY }to specify the minimum and maximum energy
for the UPPER level of the transition, to be expressed in Joules.

(ucd={\textquotedbl} phys.energy; phys.atmol.level'';
utype=''\textbf{ssldm:Line.upperLevel.energy.value''})

Energy of upper level between 3.93E-18 and 3.94E-18 Joules:

\parexample{UPPER\_LEVEL\_ENERGY=3.93E-18 3.94E-18}

\subsubsection{Temperature range}
\label{RefHeadingToc5716708}

A service \textbf{MAY} implement the parameter \textbf{TEMPERATURE} to
specify the minimum and maximum expected temperatures of the object, to
be specified in Kelvin. This parameter would be used (in particular for
theoretical spectral line databases) to sort the lines in the output
using physical models.

(ucd=''phys.temperature'';utype=''\textbf{ssldm:Line.environment.temperature.value}'')

Temperature between 10 and 50 Kelvins:

\parexample{TEMPERATURE=10 50}

\subsubsection{Einstein coefficient A range}
\label{RefHeadingToc5716709}

A service \textbf{MAY }implement a parameter \textbf{EINSTEIN\_A} to accept constraints in the transition probability by
specifying the minimum and maximum Einstein A, defined as the probability per unit time s\textsuperscript{-1} for
spontaneous emission in a bound-bound transition.

(ucd='' phys.atmol.transProb''; utype=''\textbf{ssldm:Line.einsteinA.value}'').

Transition probability between 1.1E-7 and 1.2E-7 s-1:

\parexample{EINSTEIN\_A=1.1E-7 1.2E-1}

\subsubsection{Physical process parameters}
\label{RefHeadingToc5716710}

It is possible to discriminate the result lines per physical process
that originate or modify the line. To achieve this result, and in line
with SSLDM, the following input parameters could be added.

\paragraph{Process type}

A service \textbf{MAY} implement the parameter \textbf{PROCESS\_TYPE} to
specify the physical process type responsible for the generation of the
line or for the modification of its physical properties
(utype=''\textbf{ssldm:Process.type}'').

Valid values are: ``Matter-radiation interaction'', `Matter-matter
interaction'', ``Energy shift'', ``Broadening'' as specified in the
section 3.8.1 of the Simple Spectral Line Data Model.

Search Broadening related lines:

\parexample{PROCESS\_TYPE=Broadening}

\paragraph{Process name}

A service \textbf{MAY} implement the parameter \textbf{PROCESS\_NAME} to
specify the physical process responsible for the generation of the line
or for the modification of its physical properties
(utype=''\textbf{ssldm:Process.name}''). \ There is a great variety of
possible values, so this input parameter would need to make use of the
capabilities resource, as described in related section.

Some possible values are: ``Photoionization'', ``Collisional
excitation'', ``Gravitational redshift'', ``Stark broadening'',
``Resonance broadening'', ``Van der Waals broadening'', etc... \ as
specified in the section 3.8.2 of the Simple Spectral Line Data Model.

Search lines with Stark Broadening process:

\parexample{PROCESS\_NAME=Stark Broadening}

\subsubsection{Custom query parameters}
\label{RefHeadingToc5716711}

As we saw in Section 3, there is a need to have a general mechanism for
free query parameters to filter out or sort the table result.

Both for the non-compulsory parameters and/or for the free ones, client
applications can discover whether a particular parameter is implemented
through the VOSI-capabilities operation.

Using this information, a VO client would be able to dynamically
construct a form, where this information could be inserted.

\subsection{Successful output}
\label{RefHeadingToc5716712}

The output returned by a SLAP service is a VOTable
\citep{2013ivoa.spec.0920O}, an XML table format, returned with a
MIME-type of {\textquotedbl}application/x-votable+xml{\textquotedbl}.
The table lists all the Spectral lines found in the server database that
match the query constraints.

It \textbf{MUST} contain a RESOURCE element identified with the tag
type={\textquotedbl}results{\textquotedbl} that \textbf{SHOULD} contain
a single TABLE element which contains the results of the query. The
VOTable is permitted to contain additional RESOURCE elements, but the
usage of any such elements is not defined here. If multiple resources
are present it is recommended that the query results be returned in the
first resource element.

The VOTable \textbf{MAY} contain references to other name spaces, like
SLAP, Characterization, etc. The VOTable \textbf{MUST} contain a
reference to the SSLDM namespace.

\begin{verbatim}
xmlns:ssldm="http://www.ivoa.net/xml/SimpleSpectralLineDM/v2.0"
\end{verbatim}

\subsubsection[RESOURCE element]{RESOURCE element}
\label{RefHeadingToc5716713}

The RESOURCE element contains several INFO elements storing metadata
about the request execution and the queried service, as described in the
following sections.

\paragraph{QUERY\_STATUS}

It \textbf{MUST} contain an INFO with
name={\textquotedbl}QUERY\_STATUS{\textquotedbl}. Its value attribute
MUST be set to ``OK{\textquotedbl} if the query executed successfully,
regardless of whether any matching spectral lines were found. \ If an
overflow occurs (result exceeds MAXREC) the value attribute will be set
to ``OVERFLOW''.

\paragraph{REQUEST\_COMPLETED\_TIMESTAMP}

The RESOURCE element \textbf{SHOULD} contain an INFO with
name={\textquotedbl}REQUEST\_COMPLETED\_TIMESTAMP{\textquotedbl}. Its
value attribute should contain the UNIX timestamp in seconds when the
file was created by the service.

\paragraph{SERVICE\_VERSION}

The RESOURCE element \textbf{SHOULD} contain an INFO with
name={\textquotedbl}SERVICE\_VERSION{\textquotedbl}. Its value attribute
contains the version of the data in the database on which the service is
relying, in order to follow data evolution over time. The format of this
value is managed by the data provider. It must be updated each time the
database content evolves.

\paragraph{SERVICE\_NAME}

The RESOURCE element \textbf{SHOULD} contain an INFO with
name={\textquotedbl}SERVICE\_NAME{\textquotedbl}. Its value attribute
contains the name of the service that was queried. The format of this
value is managed by the data provider.

\paragraph{REQUEST\_DESCRIPTION}

The RESOURCE element \textbf{SHOULD} contain an INFO with
name={\textquotedbl}REQUEST\_DESCRIPTION{\textquotedbl}. Its value
attribute contains a text description of the request that has been
performed to obtain the data.

\begin{lstlisting}{language=XML}
<INFO name="QUERY_STATUS" value="OK" />
<INFO name="FILE_TIMESTAMP" value="1485360332" />
<INFO name="SERVICE_VERSION" value="2017_01_25"/>
<INFO name="SERVICE_NAME" value="slap service name"/>
\end{lstlisting}

\subsubsection{The TABLE element}
\label{RefHeadingToc5716714}

\paragraph{Standard column with multiple units}

A standard column could appear multiple times with different units. When
a standard column can appear multiple times with the same utype but
different units, the column is uniquely identified by its utype and
unit.  Otherwise, a standard column is uniquely defined by its utype.

Each non standard FIELD \textbf{SHOULD} contain a utype reference to
the Simple Spectral Line Data Model whenever possible.

\paragraph{Table rows}

Each table row of represents a different spectral line available to the
client. A standard column \textbf{MUST} have a defined utype and a
defined UCD as described in the next section.

\subsection{Standard output fields}
\label{RefHeadingToc5716715}

For a given field, it is allowed to have more than one instance of it,
with different units in order to preserve the precision of the original
value prior to unit conversion in the client. If this is the case and a
default unit is specified in this document, the latter value
\textbf{MUST} be returned and other fields with the same utype should
have a different value in the unit field descriptor.

\subsubsection[Line.wavelength.value]{Line.wavelength.value}
\label{RefHeadingToc5716716}

One field \textbf{MUST} have
utype={\textquotedbl}\textbf{ssldm:Line.wavelength.value{\textquotedbl}},
with datatype={\textquotedbl}double{\textquotedbl}
unit={\textquotedbl}m{\textquotedbl} and
ucd={\textquotedbl}em.wl{\textquotedbl}, containing the wavelength in
vacuum of the transition originating the line in meters.

\subsubsection{Line.title}
\label{RefHeadingToc5716717}

Exactly one field \textbf{MUST }have
utype=''\textbf{ssldm:Line.title}'', with
datatype={\textquotedbl}char{\textquotedbl}
arraysize={\textquotedbl}*{\textquotedbl} and ucd=''meta.title'',
containing a small description identifying the line.

Note that this line title is only a short string representation to be
used in the clients for display. There is no required syntax, but it is
recommended that common species and transition notation be used when
applicable.

Examples: \verb|H I|, \verb|N III 992.873 A|

In case of corrected but unidentified lines, some examples could be:

Examples: \verb|M31 1001.784 A|, \verb|011910191 800.2 A|

\subsubsection{Line.identificationStatus}
\label{RefHeadingToc5716718}

Exactly one field \textbf{SHOULD }have
utype=''\textbf{ssldm:Line.identificationStatus}'' with
datatype={\textquotedbl}char{\textquotedbl},
arraysize={\textquotedbl}*{\textquotedbl}, containing the identification
status of the line. Possible values are: UNIDENTIFIED,
IDENTIFICATION\_UNCERTAIN, IDENTIFICATION\_PROVISIONAL, IDENTIFIED.

See Appendix \ A for details.

\subsubsection{Line.lowerLevel.element.name}
\label{RefHeadingToc5716719}

Exactly one field \textbf{SHOULD} have
utype=''\textbf{{ssldm:Line.lowerLevel.element.name}}'' with
datatype={\textquotedbl}char{\textquotedbl},
arraysize={\textquotedbl}*{\textquotedbl} and
ucd=''phys.atmol.element'', containing the name of the chemical element
source of the lower level of this line.

\subsubsection{Line.upperLevel.element.name}
\label{RefHeadingToc5716720}

Exactly one field \textbf{SHOULD} have
utype=''\textbf{ssldm:Line.upperLevel.element.name}'' with
datatype={\textquotedbl}char{\textquotedbl},
arraysize={\textquotedbl}*{\textquotedbl} and
ucd=''phys.atmol.element'', containing a name of the chemical element
for the upper level of this line. If only one of lower and upper element
name is specified, it is assumed that both are identical.

\subsubsection{Line.lowerLevel.element.inChiKey}
\label{RefHeadingToc5716721}

Exactly one field \textbf{SHOULD} have
utype=''\textbf{{ssldm:Line.lowerLevel.element.inChiKey}}'' with
datatype={\textquotedbl}char{\textquotedbl},
arraysize={\textquotedbl}*{\textquotedbl} and
ucd=''phys.atmol.element'', containing the inchikey \citep{inchi} of
the chemical element for the lower level of this line.

\subsubsection{Line.upperLevel.element.inChiKey}
\label{RefHeadingToc5716722}

Exactly one field \textbf{SHOULD} have
utype=''\textbf{{ssldm:Line.upperLevel.element.inChiKey}}'' with
datatype={\textquotedbl}char{\textquotedbl},
arraysize={\textquotedbl}*{\textquotedbl} and
ucd=''phys.atmol.element'', containing the inchikey of the chemical
element for the upper level of this line. If only one of lower and upper
element inChiKey \ is specified, it is assumed that both are
identical.

\subsubsection{Line.lowerLevel.element.inChi}
\label{RefHeadingToc5716723}

Exactly one field \textbf{MAY} have
utype=''\textbf{{ssldm:Line.lowerLevel.element.inChi}}'' with
datatype={\textquotedbl}char{\textquotedbl},
arraysize={\textquotedbl}*{\textquotedbl} and
ucd=''phys.atmol.element'', containing the inchi of the
chemical element for the lower level of this line.

\subsubsection{Line.upperLevel.element.inChi}
\label{RefHeadingToc5716724}

Exactly one field \textbf{MAY} have
utype=''\textbf{{ssldm:Line.upperLevel.element.inChi}}{'' with
datatype={\textquotedbl}char{\textquotedbl},
arraysize={\textquotedbl}*{\textquotedbl} and
ucd=''phys.atmol.element'', containing the inchi of the chemical element
for the upper level of this line. If only one of lower and upper state
element inChi is specified, it is assumed that both are identical.

\subsubsection{Line.lowerLevel.element.ionCharge}
\label{RefHeadingToc5716725}

Exactly one field \textbf{SHOULD} have
utype=''\textbf{{ssldm:Line.lowerLevel.element.ionCharge}}'' with
datatype={\textquotedbl}int{\textquotedbl} and
ucd=''phys.atmol.ionization'', containing the charge of the chemical
element origin of the lower level of the line.

\subsubsection{Line.upperLevel.element.ionCharge}
\label{RefHeadingToc5716726}

Exactly one field \textbf{SHOULD} have
utype=''\textbf{{ssldm:Line.upperLevel.element.ionCharge}}'' with
datatype={\textquotedbl}int{\textquotedbl} and
ucd=''phys.atmol.ionization'', containing the charge of the chemical
element origin of the upper level of the line. If only one of lower and
upper element ion charge is specified, it is assumed that both are
identical.

\subsubsection{Line.lowerLevel.name}
\label{RefHeadingToc5716727}

Exactly one field \textbf{SHOULD }have
utype=''\textbf{ssldm:Line.lowerLevel.name}'', with
datatype={\textquotedbl}char{\textquotedbl},
arraysize={\textquotedbl}*{\textquotedbl} {and ucd=''phys.atmol.level'',
containing a full description of the lower level of the transition
originating the line.}}

\subsubsection{Line.upperLevel.name}
\label{RefHeadingToc5716728}

Exactly one field \textbf{SHOULD} have
utype=''\textbf{ssldm:Line.upperLevel.name}'', with
datatype={\textquotedbl}char{\textquotedbl},
arraysize={\textquotedbl}*{\textquotedbl} and ucd=''phys.atmol.level'',
containing a full description of the upper level of the transition
originating the line.

\subsubsection{Line.observedWavelength.value}
\label{RefHeadingToc5716729}

Exactly one field \textbf{MAY} have
utype=''\textbf{{ssldm:Line.observedWavelength.value}}'',with
datatype={\textquotedbl}double{\textquotedbl}
unit={\textquotedbl}m{\textquotedbl} and
ucd={\textquotedbl}em.wl{\textquotedbl}, containing the observed
wavelength in the vacuum of the transition originating the line in
meters, as it was observed. This may be used by observational spectral
line databases.

\subsubsection[Query.Score]{Query.Score}
\label{RefHeadingToc5716730}

Exactly one field \textbf{MAY} have utype=''\textbf{slap:Query.Score}'',
with datatype={\textquotedbl}double{\textquotedbl}. A line with a higher
score more closely matches the query parameters.  The query response
table should normally be returned sorted in order of decreasing values
of score, with the top-scoring items at the top of the list. The details
of the heuristic used to compute the score are left to the service. This
is particularly useful for the theoretical spectral line databases
examples explained in section 3\todo{ref!}. Please note that a reference to the
relevant characterization namespace is needed in the VOTable response.

\begin{lstlisting}
xmlns:slap=http://www.ivoa.net/xml/SimpleLineAccessDM/v2.0"
\end{lstlisting}

\subsubsection{Line.lowerLevel.energy.value}
\label{RefHeadingToc5716731}

Exactly one field \textbf{MAY} have
utype=''\textbf{ssldm:Line.lowerLevel.energy.value}'', with
datatype={\textquotedbl}double{\textquotedbl}, unit=''J'' and
ucd={\textquotedbl}phys.energy;phys.atmol.level{\textquotedbl} and
exactly one field \textbf{MAY} have
utype=''\textbf{ssldm:Line.upperLevel.energy.value}'', with
datatype={\textquotedbl}double{\textquotedbl}, unit=''J'' and
ucd={\textquotedbl}phys.energy;phys.atmol.level{\textquotedbl},
describing the energy for the lower and upper levels of the transition
respectively which originates this line. The value must appear in Joules
to unify results.

It is allowed to have more than one
\textbf{Line.lowerLevel.energy.value} and
\textbf{Line.upperLevel.energy.value} fields in different units in order
to preserve the precision of the original value prior to unit
conversion. If this is the case and to get backwards compatibility with
already existing services, there \textbf{MUST} be one field with
utype=''\textbf{ssldm:Line.lowerLevel.energy.value}'' and one with
utype=\textbf{ ssldm:Line.upperlLevel.energy.value}'' where unit=''J''
in the VOTable response. Other fields with the same utype should have a
different value in the unit field descriptor.

\subsubsection{Line.lowerLevel.configuration}
\label{RefHeadingToc5716732}

Exactly one field \textbf{MAY} have
utype=''\textbf{ssldm:Line.lowerLevel.configuration}'' with
ucd={\textquotedbl}phys.atmol.configuration;{\textquotedbl} and one with
utype=''\textbf{ssldm:Line.upperLevel.configuration}'', with
ucd={\textquotedbl}phys.atmol.configuration{\textquotedbl}, \
datatype=''char{\textquotedbl}, and arraysize=''*'' describing the
electron configuration of the lower and upper levels of the line.

The format of the string representing the electron configuration is as
follows.

For atomic levels, Fe basic atomic level configuration is:

$$
\hbox{1\textit{s}\textsuperscript{2}2\textit{s}\textsuperscript{2}2\textit{p}\textsuperscript{6}3\textit{s}\textsuperscript{2}3\textit{p}\textsuperscript{6}3\textit{d}\textsuperscript{6}4\textit{s}\textsuperscript{2
}= [Ar] 3d\textsuperscript{6}4s\textsuperscript{2 }=
3d\textsuperscript{6}4s\textsuperscript{2}}$$

Where we have subtracted the closed shell configuration from the enumeration.

There is no required syntax for this string; however, it is recommended
to use a LaTeX-styled \citep{latexstyle} convention to describe sub-indexes,
super-indexes, greek characters, etc. That means the previous atomic
configuration could be serialized as string in the SLAP response in the
following way:

$$\hbox{3d\^6 4s\^2}$$

In the case of molecular levels, we usually need to use Greek symbols,
so, e.g.,

$$
\textrm{\textsuperscript{2}}\textrm{\textit{\textsuperscript{S}}}\textrm{\textsuperscript{
+ 1}}\textrm{$\Lambda $}\textrm{\textsubscript{$\Omega
$}}\textrm{ }
$$

would be serialized as

$$\hbox{\^\{2S+1\}{\textbackslash}Lambda \_\{{\textbackslash}Sigma \}}$$

See \citep{latexstyle} for a complete reference.

\subsubsection{Line.lowerLevel.quantumState}
\label{RefHeadingToc5716733}

Exactly one field \textbf{MAY} have
utype=''\textbf{ssldm:Line.lowerLevel.quantumState}'' and one with
utype=''\textbf{ssldm:Line.upperLevel. quantumState}'', with
datatype=''char{\textquotedbl} and arraysize=''*'' describing the
quantumState of the lower and upper levels in a parseable string
representation. The format must comply with the following syntax:

$$
\hbox{[label:type:numerator:denominator;label:type:numerator:denominator;
{\dots}][{\dots}]}$$

Where, for the lower level:

\begin{itemize}
\item label is a string legal value for the model component given by
\textbf{ssldm:Line.lowerLevel.quantumState.quantumNumber.label}
\item type is a string legal value for the model component given by
\textbf{ssldm:Line.lowerLevel.quantumState.quantumNumber.type}
\item numerator is a integer legal value for the model component given by
\textbf{ssldm:Line.lowerLevel.quantumState.quantumNumber.numeratorValue}
\item denominator is a integer legal value for the model component given by
\textbf{ssldm:Line.lowerLevel.quantumState.quantumNumber.denominatorValue}
\end{itemize}

(Please refer to the Simple Spectral Line Data Model for meaning and format)

Quantum Numbers are separated by the ``;'' character.

To allow for levels needing more than one quantum state to be described, the quantum states would be enclosed by square
brackets.

Example:

[J:totalAngularMomentumJ:1:1;F1:totalAngularMomemtumF:0:1;F:totalAngularMomemtum:1:1]

\subsubsection{Process.type}
\label{RefHeadingToc5716736}

One field \textbf{MAY} have utype=''\textbf{ssldm:Process.type}'' with
datatype=''char{\textquotedbl} and arraysize=''*'' \ identifying the
types of physical processes responsible for the generation of the
modification of its physical properties. As more than one value is
possible per line, the values would be separated by the ``:'' character,
in the following way:

Example: \verb|Matter-radiation interaction:Broadening|

Valid values are: ``Matter-radiation interaction'', `Matter-matter
interaction'', ``Energy shift'', ``Broadening'' (see section 3.8.1 of
Simple Spectral Data Model).

\subsubsection{Process.name}
\label{RefHeadingToc5716737}

One field \textbf{MAY} have utype=''\textbf{ssldm:Process.name}'' with
datatype=''char{\textquotedbl} and arraysize=''*'', identifying a
description of the physical processes responsible for the generation of
for the modification of its physical properties. As more than one value
is possible per line, the values would be separated by the ``:''
character, in the following way:

Example: \verb|Photoionization:Natural broadening|

This list should be in correspondence with the value of the possible
process types described in previous section. Please notice that ``:'' is
a reserved character.

\subsubsection{Line.Reference.url}
\label{RefHeadingToc5716738}

One or more fields \textbf{MAY} have
utype={\textquotedbl}\textbf{ssldm:Line.Reference.url}{\textquotedbl}
ucd={\textquotedbl}meta.bib;meta.ref.url{\textquotedbl}, with
datatype={\textquotedbl}char{\textquotedbl} and
arraysize={\textquotedbl}*{\textquotedbl}, to specify an http link that
contains a scientific publication related to the spectral line. Since
the URL will often contain \ metacharacters, the URL is normally
enclosed in an XML CDATA section
({\textless}![CDATA[...]]{\textgreater}) or otherwise encoded to escape
any embedded metacharacters.

\subsubsection{Line.Reference.bibtex}
\label{RefHeadingToc5716739}

One or more fields \textbf{MAY} have
utype={\textquotedbl}\textbf{ssldm:Line.Reference.bibtex}{\textquotedbl},
\ ucd={\textquotedbl}meta.bib{\textquotedbl} with
datatype={\textquotedbl}char{\textquotedbl} and
arraysize={\textquotedbl}*{\textquotedbl} and free names, to specify the
description of url-less references in bibtex format. Since this notation
will contain metacharacters, it must be enclosed in an XML CDATA section
({\textless}![CDATA[...]]{\textgreater}).

Example:

\begin{lstlisting}
@misc{BCDMS-1920},
  author = {Evenson, K. M.},
  title = {},
  year = {1995},
  howpublished={'private communication'}
\end{lstlisting}

\subsubsection{Additional references}
\label{RefHeadingToc5716740}

One or more fields \textbf{MAY} have
ucd={\textquotedbl}meta.ref.url{\textquotedbl} with
datatype={\textquotedbl}char{\textquotedbl} and
arraysize={\textquotedbl}*{\textquotedbl} and free names, to specify
URLs that contains extra information related to the spectral line. Same
criteria than before about the CDATA section.

\subsubsection{Observational line databases specificities}
\label{RefHeadingToc5716741}

In the case of observational line databases, some characterization
information of the observation itself could be relevant. Next, we
present some examples of possible observation-related output metadata
(note that all the following fields are in line to the ones described in
the SSAP specification.

\begin{itemize}
\item Exactly one field \textbf{MAY} have
utype=''\textbf{Target.Name}'', with
datatype={\textquotedbl}char{\textquotedbl}, and arraysize=''*''
containing a short string identifying the observed astronomical object,
suitable for input to a name resolver.
\item Exactly one field \textbf{MAY} have
utype=''\textbf{char:SpatialAxis.Coverage.Location.Value}'', with
datatype={\textquotedbl}double{\textquotedbl}, arraysize=''*'' and
ucd={\textquotedbl}pos{\textquotedbl}, containing the observation
position of the observation in the format: ra dec, white space separated
and both in deg. Please note that a reference to the relevant
characterization namespace is needed in the VOTable response,
\verb|xmlns:char="http://www.ivoa.net/xml/Characterisation/v1.11"|
\item Exactly one field \textbf{MAY} have
utype=''\textbf{char:TimeAxis.Coverage.Bounds.Start}'', with
datatype={\textquotedbl}char{\textquotedbl}, arraysize=''*'' and
ucd={\textquotedbl} time.start;obs.exposure{\textquotedbl}, containing
the start time for the observation in MJD with units of days. Please
note that a reference to the relevant characterization namespace is
needed in the VOTable response.

\item Exactly one field \textbf{MAY} have
utype=''\textbf{char:TimeAxis.Coverage.Bounds.Stop}'', with
datatype={\textquotedbl}char{\textquotedbl}, arraysize=''*'' and
ucd={\textquotedbl} time.stop;obs.exposure{\textquotedbl}, containing
the end time for the observation in MJD with units of days. Please note
that a reference to the relevant characterization namespace is needed in
the VOTable response.
\end{itemize}

\subsection{Non-standard output fields}
\label{RefHeadingToc5716742}

In many occasions, extra scientifically interesting parameters may be
added to the output. Implementers are encouraged to add descriptions and
UCDs to the return fields to clarify the meaning of this information and
utypes to the Line Data Model or other existing IVOA Data Model,
whenever possible.


\section{\{species\} resource}
\label{RefHeadingToc5716743}

A simple spectral service \textbf{MAY} support a resource listing all
the species for which lines are available. This operation returns the
complete list of the chemical elements available in the service. The
output returned by the service is a VOTable, an XML table format,
returned with a MIME-type of
{\textquotedbl}application/x-votable+xml{\textquotedbl}.

\subsection{Parameters}
\label{RefHeadingToc5716744}

\subsubsection{RESPONSEFORMAT}
\label{RefHeadingToc5716745}

The RESPONSEFORMAT parameter is defined in a similar way to 6.1.2 for
the \{lines\} resource.

\subsection{Successful output}
\label{RefHeadingToc5716746}

The global metadata returned by a SLAP service for the \{species\}
resource are identical to the output of the \{lines\} resource (see
section 6.3).

\subsection{Standard output fields}
\label{RefHeadingToc5716747}

\subsubsection{Species.name}
\label{RefHeadingToc5716748}

Exactly one field \textbf{MUST} have
utype={\textquotedbl}\textbf{ssldm:Species.name}{\textquotedbl},
name={\textquotedbl}SPECIES\_NAME{\textquotedbl},
ucd={\textquotedbl}phys.atmol.element{\textquotedbl},
datatype={\textquotedbl}char{\textquotedbl} and
arraysize={\textquotedbl}*{\textquotedbl}, containing the name of a
chemical element for which at least one spectral line is available in
the service.

\subsubsection{Species.type}
\label{RefHeadingToc5716749}

{Exactly one field }\textbf{{MAY }}{have
utype=''}\textbf{{ssldm:Species.type}}{'',
name={\textquotedbl}SPECIES\_TYPE{\textquotedbl} with
datatype={\textquotedbl}char{\textquotedbl} and
arraysize={\textquotedbl}*{\textquotedbl}. Possible values are Atom,
Molecule and Particle.

\subsubsection[Species.ionCharge]{Species.ionCharge}
\label{RefHeadingToc5716750}

Exactly one field \textbf{SHOULD} have
utype={\textquotedbl}\textbf{ssldm:Species.ionCharge}{\textquotedbl},
name={\textquotedbl}ION\_CHARGE{\textquotedbl},
ucd={\textquotedbl}phys.atmol.ionization{\textquotedbl} and
datatype={\textquotedbl}int'' containing the charge of the considered
chemical species.

\subsubsection{Species.inChiKey}
\label{RefHeadingToc5716751}

Exactly one field \textbf{{MAY} have
utype={\textquotedbl}}\textbf{{ssldm:Species.inChiKey}}{{\textquotedbl},
name={\textquotedbl}INCHIKEY{\textquotedbl} with
datatype={\textquotedbl}char{\textquotedbl} and
arraysize={\textquotedbl}*{\textquotedbl}, containing the inchikey of
the chemical element.

\subsubsection{Species.inChi}
\label{RefHeadingToc5716752}

Exactly one field \textbf{MAY} have
utype={\textquotedbl}}\textbf{{ssldm:Species.inChi}}{{\textquotedbl},
name={\textquotedbl}INCHI{\textquotedbl} with
datatype={\textquotedbl}char{\textquotedbl} and
arraysize={\textquotedbl}*{\textquotedbl}, containing the inchi of the
chemical element.

\subsection{Example}
\label{RefHeadingToc5716753}

\todo{Translate}

\section{VOSI-availability}
\label{RefHeadingToc5716754}

A SLAP service \textbf{MUST} provide a VOSI-availability resource.

\nolinkurl{http://example.com/slap2/availability}}

\section{VOSI-capabilities }
\label{RefHeadingToc5716755}

A compliant simple spectral line service \textbf{MUST} provide a
\{capabilities\} resource used to query the service metadata:

\nolinkurl{http://example.com/slap2/capabilities}

This resource will return the capability metadata as described in IVOA
Support Interfaces document. The DALI-sync resources will
be identified through the standardID attribute:

\begin{inlinetable}
\begin{tabular}{p{4.8780003cm}p{8.736cm}}
\sptablerule
\textbf{DALI-sync resource} &
\textbf{standardID}\\
\sptablerule
\{lines\} &
{ivo://ivoa.net/std/SLAP\#lines-2.0} \\
\{species\} &
{ivo://ivoa.net/std/SLAP\#species-2.0}\
\\sptablerule
\end{tabular}
\end{inlinetable}

An example of such a document could be:

\begin{lstlisting}[language=XML]
<vosi:capabilities
  xmlns:vosi="http://www.ivoa.net/xml/VOSICapabilities/v1.0"
  xmlns:xsi="http://www.w3.org/2001/XMLSchema-instance"
  xmlns:vs="http://www.ivoa.net/xml/VODataService/v1.1">
  <capability>
    <interface xsi:type="vs:ParamHTTP" role="std">
      <accessURL use="full">
        http://example.com/slap2/capabilities
     </accessURL>
    </interface>
  </capability>
  <capability>
    <interface xsi:type="vs:ParamHTTP" role="std">
      <accessURL use="full">
        http://example.com/slap2/availability
      </accessURL>
    </interface>
  </capability>
  <capability standardID="ivo://ivoa.net/std/SLAP#species-2.0">
    <interface xsi:type="vs:ParamHTTP" role="std" version="2.0">
      <accessURL>
        http://example.com/slap2/species
      </accessURL>
    </interface>
  </capability>
  <capability standardID="ivo://ivoa.net/std/SLAP#lines-2.0">
    <interface role="std" version="2.0" xsi:type="vs:ParamHTTP">
      <accessURL use="base">http://example.com/slap2/lines?</accessURL>
      <queryType>GET</queryType>
      <resultType>application/x-votable+xml</resultType>
      <!-- mandatory parameter -->
      <param use="required" std="true">
        <name>wavelength</name>
        <description>wavelength in the vacuum of the transition
          originating in the line</description>
        <unit>m</unit>
        <ucd>em.wl</ucd>
        <utype>ssldm:Line.wavelength.value</utype>
        <dataType arraysize="2">real</dataType>
      </param>
      <!-- non-compulsory parameters -->
      <param std="true">
        <name>chemical_element</name>
        <description>Name of the searched chemical element</description>
        <ucd>phys.atmol.element</ucd>
        <utype>ssldm:Line.lowerLevel.element.name</utype>
        <dataType arraysize="*">char</dataType>
      </param>
      <param std="true">
        <name>ion_charge</name>
        <description>Minimum and maximum charge of an ion</description>
        <ucd>phys.atmol.ionization</ucd>
        <utype>ssldm:Line.lowerLevel.element.ionCharge</utype>
        <dataType arraysize="2">integer</dataType>
      </param>
      <!-- custom service parameter -->
      <param std="false">
        <name>wavenumber</name>
        <description>Minimum and maximum wavenumber of the line</description>
        <ucd>em.wavenumber</ucd>
        <utype>ssldm:Line.waveNumber</utype>
        <dataType arraysize="2">integer</dataType>
      </param>
    </interface>
  </capability>
</vosi:capabilities>
\end{lstlisting}

\section{Registration and discovering of SLAP resources}
\label{RefHeadingToc5716756}

Users can discover a service instance through an IVOA-compliant
registry. \ The description is recognized as an instance of a SLAP
service when it complies with the SLAP VOResource metadata extension
schema \citep{2017ivoa.spec.0530P}.

\appendix

\section{SLAP services for uncorrected and unidentified lines}
\label{RefHeadingToc5716757}

One important constraint for observational line databases in the SLAP
definition is to limit the lines in the SLAP output to corrected lines,
i.e. only the lines that have been corrected from wavelength shifts
(including not only the astrophysical redshift, but also any process
that could produce a line peak displacement). \ (Note that unidentified
lines can be distributed as long as these lines could be corrected in
wavelength. See observational ``\textbf{ssldm:Line.title}'' recommended
syntax.)

The reason for that is, if a user wants to compare two different SLAP
service outputs or to compare one SLAP service output to one SSAP
service output, the data comparison requires previous correction. This
correction is often quite complex and it should be done by the data
providers, the ones who have the knowledge and expertise to correct the
data properly, and not by a client application, as this process is prone
to scientific errors.

Even if a SLAP-like service cannot be registered in the IVOA as standard
SLAPs because it provides only uncorrected lines, the creation of
services for internal project consumption could be desirable.

If a service is able to provide both corrected and uncorrected lines,
the service could be registered as soon as a default call to the service
only returns the corrected lines as output. The following optional
parameter could be used to select other types of lines.

\begin{itemize}
\item {\bfseries CORRECTION\_STATUS}
A service MAY have a search parameter called CORRECTION\_STATUS.
This parameter would constraint the search to lines with a certain level
of identification. As specified in SSLDM the possible values are:

\begin{itemize}
\item UNCORRECTED
\item CORRECTED
\end{itemize}

And the default value MUST be ``CORRECTED''
(ucd=``em.line;meta.id.cross''; utype=''ssldm:Line.correctionStatus'')

\item {\bfseries IDENTIFICATION\_STATUS}
A service \textbf{MAY} have a search parameter called
IDENTIFICATION\_STATUS. This parameter would constrain the search to
lines with a certain level of identification. As specified in SSLDM the
possible values are:

\begin{itemize}
\item UNIDENTIFIED
\item IDENTIFICATION\_UNCERTAIN
\item IDENTIFICATION\_PROVISIONAL
\item IDENTIFIED
\end{itemize}

(ucd=``em.line;meta.id.cross'';
utype=''\textbf{ssldm:Line.identificationStatus''})

For theoretical line databases, identified will be synonymous to predicted.
\end{itemize}

For the output:

\begin{itemize}
\item Exactly one field \textbf{MAY} have
utype=''ssldm:Line.correctionStatus'', with
datatype={\textquotedbl}char{\textquotedbl} and
ucd=``em.wl.central;meta.id.cross{\textquotedbl}, describing if the
``ldm:Line.wavelength'' has been calculated.

Please note that for uncorrected lines the compulsory field with
utype=''\textbf{ldm:Line.wavelength}'' (corrected wavelength in vacuum)
will contain either the observed wavelength for ``Uncorrected'' or a
provisional assignment value.

\item Exactly one field \textbf{MAY} have
utype=''\textbf{ssldm:Line.identificationStatus}'', with
datatype={\textquotedbl}char{\textquotedbl} and
ucd=``em.line;meta.id.cross{\textquotedbl}, describing the
identification status of the line.

\end{itemize}


\section{SLAP valid response example}

\begin{lstlisting}[language=XML]
<VOTABLE xmlns:xsi="http://www.w3.org/2001/XMLSchema-instance" xmlns:ssldm="http://www.ivoa.net/xml/SimpleSpectralLineDM/v2.0" xsi:noNamespaceSchemaLocation="xmlns:http://www.ivoa.net/xml/VOTable/VOTable-1.3.xsd" version="1.3">
  <RESOURCE type="results">
    <INFO name="QUERY_STATUS" value="OK"/>
    <INFO name="REQUEST_COMPLETED_TIMESTAMP" value="1487063789"/>
    <INFO name="SERVICE_VERSION" value="2017_01_25"/>
    <INFO name="SERVICE_NAME" value=" European Space Astronomy Centre - Simple Line Access Protocol (SLAP)"/>
    <TABLE>
      <FIELD ucd="meta.id;obs" name="OBSNO" datatype="char" arraysize="*"/>
      <FIELD ucd="em.wl" name="WAVELENGTH" utype="ssldm:Line.wavelength.value" datatype="double" unit="m"/>
      <FIELD ucd="em.freq" name="FREQUENCY" utype=" ssldm:Line.frequency.value" datatype="double"/>
      <FIELD ucd="em.wavenumber" name="WAVE_NUMBER" utype="ssldm:Line.wavenumber.value" datatype="double"/>
      <FIELD ucd="spect.line.width" name="LINEWIDTH" utype="ssldm:Line.observedBroadeningCoefficient.value" datatype="double"/>
      <FIELD ucd="spect.line.width;meta.unit" name="LINEWIDTH_UNIT" datatype="char" arraysize="*"/>
      <FIELD ucd="meta.bib" name="ADS_CODE" datatype="char" arraysize="*"/>
      <FIELD ucd="em.line" name="IDENTIFICATION" utype="ldm:Line.title" datatype="char" arraysize="*"/>
      <FIELD ucd="phys.atmol.transition" name="TRANSITION" datatype="char" arraysize="*"/>
      <FIELD name="LINE_TYPE" datatype="char" arraysize="*"/>
      <FIELD ucd="spect.line;phot.flux" name="FLUX" utype="ssldm:Line.observedFlux.value" datatype="double"/>
      <FIELD ucd="spect.line.intensity" name="PEAK_INTENSITY" utype="ssldm:Line.observedIntensity.value" datatype="double"/>
      <FIELD ucd="spect.line.intensity;stat.snr" name="SIGNAL_TO_NOISE" utype=" ssldm:Line.significanceOfDetection.value" datatype="double"/>
      <DATA>
        <TABLEDATA>
          <TR>
            <TD>116003190</TD>
            <TD>8.03E-6</TD>
            <TD>37333.748443</TD>
            <TD>1245.330012</TD>
            <TD>0.008580</TD>
            <TD>micron</TD>
            <TD>2001ApJ...552..544F</TD>
            <TD>H2</TD>
            <TD>0-0 S(4) </TD>
            <TD>L</TD>
            <TD>6.800000088194428E-16</TD>
            <TD> </TD>
            <TD> </TD>
          </TR>
          ...more lines data...
        </TABLEDATA>
      </DATA>
    </TABLE>
  </RESOURCE>
</VOTABLE>
\end{lstlisting}

\section{SLAP Data Model summary}

\begin{supertabular}{|p{6.976cm}|p{2.023cm}|p{4.88cm}|p{1.388cm}|p{1.987cm}|}
\sptablerule
 \textbf{UTYPE} &
 \textbf{UCD} &
 \textbf{Description} &
 \textbf{Type}&
 \textbf{Size}\\
 \sptablerule
\bfseries ssldm:Line.title (REQUIRED)

~
 &
 em.line &
 small description identifying the line

~

~
 &
 char &
 *\\
\bfseries ssldm:Line.wavelength.value

\bfseries (REQUIRED) &
 em.wl &
 wavelength in the vacuum of the transition originating the line in meters.

~
 &
 double &
~
\\
\bfseries ssldm:Line.lowerLevel.element.name &
 phys.atmol.element; &
 Name of the chemical element source of this line &
 char &
 *\\
\bfseries ssldm:Line.upperLevel.element.name &
 phys.atmol.element; &
 Name of the resulting chemical element for this line &
 char &
 *\\
\bfseries ssldm:Line.lowerLeve.element.type &
~
 &
 Type of the chemical element source of this line (atom or molecule) &
 char &
 *\\
\bfseries ssldm:Line.upperLevel.element.type &
~
 &
 Type of the resulting chemical for this line (atom or molecule) &
 char &
 *\\
\bfseries ssldm:Line.lowerLevel.element.inChiKey &
 phys.atmol.element; &
 InChiKey of the chemical element source of this line &
 char &
 *\\
\bfseries ssldm:Line.upperLevel.element.inChiKey &
 phys.atmol.element; &
 InChiKey of the resulting chemical element for this line &
 char &
 *\\
\bfseries ssldm:Line.lowerLevel.element.inChi &
 phys.atmol.element; &
 InChi of the chemical element source of this line &
 char &
 *\\
\bfseries ssldm:Line.upperLevel.element.inChi &
 phys.atmol.element; &
 InChi of the resulting chemical element for this line &
 char &
 *\\
\bfseries ssldm:Line.lowerLevel.element.ionCharge &
~
 &
 Charge of the chemical element source of this line &
 integer &
~
\\
\bfseries ssldm:Line.upperLevel.element.ionCharge &
~
 &
 Charge of the resulting chemical element for this line &
 integer &
~
\\
\bfseries ssldm:Line.lowerLevel.energy.value

~
 &
 phys.energy; phys.atmol.level &
 energy for the LOWER level of the transition &
 double &
~
\\
\bfseries ssldm:Line.upperLevel.energy.value

~
 &
 phys.energy; phys.atmol.level &
 energy for the UPPER level of the transition &
 double &
~
\\
\bfseries ssldm:Line.environment.temperature.value

~

~
 &
 phys.temperature &
 expected temperature of the object &
 double &
~
\\
\bfseries ssldm:Line.einsteinA.value

~
 &
 phys.atmol.transProb &
 Einstein A coefficient, probability per unit time s\textsuperscript{-1} for
spontaneous emission in a bound-bound transition &
 double &
~
\\
\bfseries ssldm:Process.type

~

~
 &
~
 &
 physical process type responsible for the generation of the line or for the
modification of its physical properties &
 char &
 *\\
\bfseries ssldm:Process.name

~
 &
~
 &
 physical process exact description responsible for the generation of the line or for
the modification of its physical properties &
 char &
 *\\
\bfseries ssldm:Line.identificationStatus &
~
 &
 identification status of the line &
 char &
 *\\
\bfseries ssldm:Line.lowerLevel.name &
 phys.atmol.level &
 description of the lower level of the transition originating the line &
 char &
 *\\
\bfseries ssldm:Line.upperLevel.name &
 phys.atmol.level &
 description of the upper level of the transition originating the line &
 char &
 *\\
\bfseries ssldm:Line.observedWavelength.value &
 em.wl &
 observed wavelength in the vacuum of the transition originating the line in meters &
 double &
~
\\
\bfseries slap:Query.Score &
~
 &
 Line ranking ``more closely matches the query parameters'' &
 double &
~
\\
\bfseries ssldm:Line.lowerLevel.configuration &
 phys.atmol.configuration &
 Description of the electron configuration of the lower level of the line &
 char &
 *

~
\\
\bfseries ssldm:Line.upperLevel.configuration &
 phys.atmol.configuration &
 describing the electron configuration of the upper level of the line &
 char &
 *\\
\bfseries ssldm:Line.lowerLevel.quantumState &
~
 &
 Description of the quantum state of the lower level in a parseable string
representation &
 char &
 *\\
\bfseries ssldm:Line.upperLevel. quantumState &
~
 &
 Description of the quantum state of the upper level in a parseable string
representation &
 char &
 *\\
\bfseries ssldm:Line.Reference.url &
 meta.bib;meta.ref.url &
 A http link that contains a scientific publication related to the spectral line &
 char &
 *\\
\bfseries ssldm:Line.Reference.bibtex &
 meta.bib &
 Description of a reference in bibtex format when no url is available &
 char &
 *\\
\bfseries Target.Name &
 meta.id;src &
 short string identifying the observed astronomical object, suitable for input to a
name resolver &
 char &
 *\\
\bfseries char:SpatialAxis.Coverage.Location.Value &
 Pos &
 observation position of the observation in the format: ra dec, white space separated
and both in deg &
 double  &
 *\\
\bfseries char:TimeAxis.Coverage.Bounds.Start &
 time.start;obs.exposure &
 start time for the observation in MJD with units of days &
 char &
 *\\
\bfseries char:TimeAxis.Coverage.Bounds.Stop &
 time.stop;obs.exposure &
 end time for the observation in MJD with units of days &
 char &
 *\\
\bfseries ssldm:Species.name (REQUIRED)

~
 &
 em.line &
 small description identifying the line

~

~
 &
 char &
 *\\
\bfseries ssldm:Species.ionCharge &
~
 &
 Charge of the chemical element source of this line &
 integer &
~
\\
\bfseries ssldm:Species.type &
~
 &
 Type of considered species, possible values are Atom, Molecule and Particle &
 char &
 *\\
 \textbf{ssldm:Species. \textbf{{inChiKey}}} &
 phys.atmol.element; &
 InChiKey of the chemical element source of this line &
 char &
 *\\
 \textbf{ssldm:\textbf{{ssldm:Species.inChi}}} &
 phys.atmol.element; &
 InChi of the chemical element source of this line &
 char &
 *\\
\end{supertabular}


\appendix
\section{Changes from Previous Versions}

Changes from version 1\todo{fill in}
% these would be subsections "Changes from v. WD-..."
% Use itemize environments.

% NOTE: IVOA recommendations must be cited from docrepo rather than ivoabib
% (REC entries there are for legacy documents only)
\bibliography{ivoatex/ivoabib,ivoatex/docrepo,local}

\end{document}
